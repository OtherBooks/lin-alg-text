\begin{frame}{\ref{s:linear systems}.\ref{ss:ge} executive summary}
\footnotesize
\alert{Definitions}: leading 1's, row echelon form, reduced row echelon form, free variables, leading variables, general solution. 
\bspace
\alert{Procedures}: Gaussian elimination, Gauss-Jordan elimination, solving a linear system of equations (=Gaussian elimination+assigning free vars. parameter names+solving for leading variables in terms of free ones.)
\bspace
\alert{Theorems}: let $A$ be a row reduced form of the augmented matrix corr. to a given linear system. 

The system of equations is inconsistent iff the last column of $A$ has a leading 1. 

If the system is consistent, then it has either exactly one solution, or infinitely many solutions. The former occurs when there are no free variables. The latter occurs when there is a free variable. 

A consistent system with more unknowns than equations has infinitely many solutions. 

A homogenous system is always consistent. 

A homogenous system with more unknowns than equations has infinitely many solutions. 



\end{frame}
\begin{frame}{Strategy for solving systems of equations}
Given system 
\[
\eqsys,
\]
represent it with the augmented matrix
\[
\augmatrix,
\]
then perform row operations to get a new, simpler matrix whose corresponding system is easier to solve. 
\end{frame}
\begin{frame}\footnotesize
We wish to 
\bb[a]
\ii Give a deterministic algorithm guaranteed to reduce a given matrix $A$ to a well-defined simpler matrix $B$. 
\ii Give a recipe for writing down all the solutions (if any) to the new system of equations corresponding to $B$. 
\ee
\pause Here is our mathematical stand-in for the notion of a ``simple" matrix.
\begin{definition}A matrix is in {\bf row echelon form} if 
\bb
\pause\ii in any nonzero row the first (i.e., leftmost) nonzero entry is a 1 (called the {\bf leading 1} of the row);
\pause\ii all zero rows are grouped at the bottom of the matrix;
\pause\ii given any two nonzero rows, the leading 1 of the lower row occurs strictly to the right of the leading 1 of the upper row (``staircase pattern"). 
\ee 
\pause A matrix is in {\bf reduced row echelon form} if in addition to (1)-(3) it also satisfies:
\bb
\ii[4.] each column containing a leading 1 has zeros everywhere else. 
\ee
\end{definition}
\end{frame}
\begin{frame}{(Reduced) row echelon form}\footnotesize
In practice to decide whether a matrix is in in (reduced) row echelon form, follow this flow chart:
\bb
\ii First see whether all zero rows are at the bottom. 
\ii For each nonzero row, determine whether the first nonzero entry is a 1, and put a box around it. 
\ii Make sure your boxes make a staircase pattern.  
\ii (For reduced row echelon form only.) Decide whether each column with a box has 0's everywhere else. 
\ee
\pause
\alert{Example.}
\[
\begin{bmatrix}
\boxed{1}&0&0&-3&{\color{red}-7}\\
0&0&\boxed{1}&2&0\\
0&0&0&0&\boxed{1}\\
0&0&0&0&0\\
0&0&0&0&0
\end{bmatrix}
\]
\pause
The matrix is in row echelon form as the boxed leading 1's make a staircase pattern. 
\bpause
The matrix is NOT in reduced row echelon form as the last column has a nonzero entry above it. \end{frame}
\begin{frame}{Gaussian elimination}

{\bf Gaussian elimination} is an algorithm which, given a matrix $A$, transforms it one row operation at a time to a matrix $B$ in row echelon form.  
\begin{description}
\item[Step 1]Find leftmost nonzero column and use row swap operation to move nonzero entry to top.
\item[Step 2] Use scaling operation to make this top entry a leading 1.
\item[Step 3] Add multiples of top row to lower rows to clear entries below the leading 1. 
\item[Step 4] Cover top row and begin again with Step 1 applied to remaining matrix. Continue until matrix is in row echelon form. 
\end{description}
\pause The next slide presents a detailed example of this procedure. 

\alert{Use this as a model when doing your own examples and exercises, including the naming of row operations used}.  
\end{frame}
\begin{frame}{Example}
\footnotesize
\begin{eqnarray*}
\scriptsize
\begin{bmatrix}
0&0&-2&0&7&12\\
2&4&-10&6&12&28\\
2&4&-5&6&-5&-1
\end{bmatrix}
\pause&\xrightarrow[\hspace{35pt}]{r_1\leftrightarrow r_2}&
\begin{bmatrix}
2&4&-10&6&12&28\\
0&0&-2&0&7&12\\
2&4&-5&6&-5&-1
\end{bmatrix}
\\
\pause&\xrightarrow[\hspace{35pt}]{\frac{1}{2}r_1}&
\begin{bmatrix}
1&2&-5&3&6&14\\
0&0&-2&0&7&12\\
2&4&-5&6&-5&-1
\end{bmatrix}
\\
\pause \alert{\text{(now done with first row)}}&\xrightarrow[\hspace{35pt}]{r_3-2r_1}&
\begin{bmatrix}
1&2&-5&3&6&14\\
0&0&-2&0&7&12\\
0&0&5&0&-17&-29
\end{bmatrix}
\\
\pause&\xrightarrow[\hspace{35pt}]{-\frac{1}{2}r_2}&
\begin{bmatrix}
1&2&-5&3&6&14\\
0&0&1&0&-7/2&-6\\
0&0&5&0&-17&-29
\end{bmatrix}
\\
\pause \alert{\text{(now done with 2nd row)}}&\xrightarrow[\hspace{35pt}]{r_3+(-5)r_2}&
\begin{bmatrix}
1&2&-5&3&6&14\\
0&0&1&0&-7/2&-6\\
0&0&0&0&1/2&1
\end{bmatrix}
\\
\pause&\xrightarrow[\hspace{35pt}]{2r_3}&
\begin{bmatrix}
\boxed{1}&2&-5&3&6&14\\
0&0&\boxed{1}&0&-7/2&-6\\
0&0&0&0&\boxed{1}&2
\end{bmatrix}
\end{eqnarray*}

\end{frame}
\begin{frame}{Gauss-Jordan elimination}\footnotesize 
The procedure for reducing a matrix all the way to \alert{reduced row echelon} is called {\bf Gauss-Jordan elimination}. \pause To do this, first perform Gaussian elimination to reduce $A$ to row echelon form, then use row operations to clear the entries \alert{above} leading 1's, starting with the rightmost and working backward. 
\bpause
Continuing our previous example:
\begin{eqnarray*}
\begin{bmatrix}
\boxed{1}&2&-5&3&6&14\\
0&0&\boxed{1}&0&-7/2&-6\\
0&0&0&0&\boxed{1}&2
\end{bmatrix}.
\pause&\xrightarrow[\hspace{35pt}]{r_2+7/2r_3}&
\begin{bmatrix}
\boxed{1}&2&-5&3&6&14\\
0&0&\boxed{1}&0&0&1\\
0&0&0&0&\boxed{1}&2
\end{bmatrix}.
\\
\pause&\xrightarrow[\hspace{35pt}]{r_1-6r_3}&
\begin{bmatrix}
\boxed{1}&2&-5&3&0&2\\
0&0&\boxed{1}&0&0&1\\
0&0&0&0&\boxed{1}&2
\end{bmatrix}.
\\
\pause&\xrightarrow[\hspace{35pt}]{r_1+5r_2}&
\begin{bmatrix}
\boxed{1}&2&0&3&0&7\\
0&0&\boxed{1}&0&0&1\\
0&0&0&0&\boxed{1}&2
\end{bmatrix}.
\end{eqnarray*}
\end{frame}
\begin{frame}
\frametitle{Solving a reduced system}
\footnotesize
It remains now to solve the resulting system corresponding to our matrix in row echelon (or reduced row echelon) form.
\bpause
Since all reduced row echelon matrices are row echelon matrices, it will suffice to outline how to solve the latter type of systems. 
\bpause Let's continue with our example: 
\begin{eqnarray*}
& & \begin{linsys}{5}
 & & & &-2x_3& & &+&7x_5&=&12\\
2x_1 & +&4x_2 & -&10x_3&+ &6x_4 &+&12x_5&=&28\\
2x_1 & +&4x_2 & -&5x_3&+ &6x_4 &-&5x_5&=&-1
\end{linsys}
\\
&\xrightarrow[]{\text{aug. mat.}}&
\begin{bmatrix}
0&0&-2&0&7&12\\
2&4&-10&6&12&28\\
2&4&-5&6&-5&-1
\end{bmatrix}\\
\pause &\xrightarrow[]{\text{row op.'s}}&
\begin{bmatrix}
\boxed{1}&2&-5&3&6&14\\
0&0&\boxed{1}&0&-7/2&-6\\
0&0&0&0&\boxed{1}&2
\end{bmatrix}\\
\pause&\xrightarrow[]{\text{new sys}}&
\begin{linsys}{5}
\boxed{x_1}&+ &2x_2&- &5x_3 &+&3x_4& +&6x_5&=&14\\
&& & &\boxed{x_3}&& &-&7/2x_5&=&-6\\
 & & & &&& &&\boxed{x_5}&=&2
\end{linsys}
\end{eqnarray*} 
\end{frame}
\begin{frame}{Solving a reduced system}\footnotesize
\footnotesize
\[
\begin{linsys}{5}
\boxed{x_1}&+ &2x_2&- &5x_3 &+&3x_4& +&6x_5&=&14\\
&& & &\boxed{x_3}&& &-&7/2x_5&=&-6\\
 & & & &&& &&\boxed{x_5}&=&2
\end{linsys}
\]
So how do we get \alert{ALL} solutions, and describe them in a useful way? Here's how
\bb
\pause\ii Divide the variables $x_1, x_2, \dots, x_5$ into {\bf leading variables} and {\bf free variables}:
\pause a leading variable is one corresponding to a column of the row reduced matrix that contains a leading 1;
\pause a free variable is any variable that is not a leading one. 
\bpause In our case, $x_1, x_3$, and $x_5$ are leading, and the rest, $x_2, x_4$ are free. 
\pause\ii  The free variables act as {\bf parameters} for our solutions. They range independently over all reals. Give them parameter names: $x_2=s$, $x_4=t$. 
\pause\ii Solve for the remaining leading variables in terms of the parameters $s, t$, starting from the rightmost leading variable and working backwards (``backwards substitution"). 
\pause We find $x_5=2$, $x_3=-6+7=1$, $x_1=7-2s-3t$. 
\ee
\pause Thus the {\bf general solution} of the system is  $(x_1,x_2,\dots,x_5)=(7-2s-3t, s, 1, t, 2)$, where $s,t$ can be any real numbers. 
\end{frame}
\begin{frame}{Solving a reduced system}
\footnotesize
\[
\begin{linsys}{5}
\boxed{x_1}&+ &2x_2&- &5x_3 &+&3x_4& +&6x_5&=&14\\
&& & &\boxed{x_3}&& &-&7/2x_5&=&-6\\
 & & & &&& &&\boxed{x_5}&=&2
\end{linsys}
\]
Some comments about how we express the solutions to a system. 
\bspace 
I described the \alert{general solution} in \alert{tuple form}:
\[
(x_1,x_2,\dots,x_5)=(5-2s-3t, s, 1, t, 2), \  \text{ $s, t$ any real numbers}.
\]
Alternatively we can give \alert{parametric equations} for each variable separately:
\[
\begin{array}{lcl}
x_1=7-2s-3t\\
x_2=s\\
x_3=1& &s,t\in\R \\
x_4=t\\
x_5=2&&
\end{array}
\]
\pause Of course we can also just write down the \alert{set of all solutions}:
\[
S=\Bigl\{(7-2s-3t, s, 1, t, 2) \colon s,t\in\R\Bigr\}.
\]
\pause Get used to using all three forms of answers. 
\end{frame}
\begin{frame}{Inconsistent system}
\footnotesize
It seems we have a surefire way of writing down general solution of any linear system. 
\bpause However, we've glossed over the situation where a linear system has \alert{no solutions}. This is called an {\bf inconsistent} system. 

\bpause When can this happen? As always, it all depends on the shape of the row echelon matrix we get after Gaussian elimination. Here is a row echelon matrix whose corresponding linear system has no solutions:
\[
\begin{bmatrix}
\boxed{1}&0&2&3\\
0&0&\boxed{1}&1\\
0&0&0&\boxed{1}
\end{bmatrix}
\]
\pause The last equation of the corresponding linear system would read $0x_1+0x_2+0x_3=1$, or $0=1$. Since this is impossible, there is no solution to the system. 
\bpause We can express this succinctly in terms of leading 1's. 
\begin{theorem}A linear system is inconsistent if and only if its augmented matrix reduces to a row echelon matrix with a leading 1 in the \alert{last column}. 
\end{theorem}
\end{frame}
\begin{frame}{Solutions to linear systems}
\bpause
We are now in a position to fully describe what solution sets to linear systems look like qualitatively. 

\pause 
\begin{theorem}[Gaussian elimination theorem]
Fix a linear system in $n$ unknowns $x_1,x_2, \dots, x_n$, let $A$ be its corresponding augmented matrix, and let $U$ be the row echelon matrix we get after applying Gaussian elimination to $A$. 
\bb
\pause\ii The system is inconsistent if and only if the \alert{last column} of $U$ has a leading 1. In this case there are $\boxed{0}$ solutions. 
\pause\ii Assume the system is consistent. 
\bb
\pause\ii If all variables are leading variables, then after back substituting, we find there is a unique solution. That is, in this case there is $\boxed{1}$ solution.  
\pause\ii If there is a free variable, say $x_i$, then since we can set $x_i=t$ for any $t$, there are $\boxed{\infty}$-many solutions
\ee   
\ee
\pause In particular, we see there are only 3 choices for the \alert{number of solutions} to a linear system: 0, 1, or $\infty$-many! \end{theorem}
\end{frame}
\begin{frame}{Solutions to homogenous systems}

Consider the special case of a homogenous system
\[
\homsys
\]
\pause Observe that $(x_1,x_2,\dots, x_n)=(0,0,\dots, 0)$ is a solution to this system. Thus a homogenous system is always consistent! \bpause (Alternatively, convince your self that the corresponding row echelon matrix $U$ for the system will have no leading 1's in the last column.) 
\end{frame}
\begin{frame}{Solutions to homogenous systems}
Since a homogenous system is always consistent, the theorem from before boils down to the following: 
\begin{corollary}
Fix a \alert{homogenous} linear system in $n$ variables. There are two possibilities:
\bb
\ii if all the variables are leading variables, then the system has a unique solution (i.e., $\boxed{1}$ solution);
\ii if there is a free variable, then the system has $\boxed{\infty}$-many solutions.  
\ee
\end{corollary}
\end{frame}
\begin{frame}{Solutions to homogenous systems}
Furthermore, we have:
\begin{corollary}
Fix a homogenous linear system. If the system has more variables than equations, then there are $\infty$-many solutions. 
\end{corollary}
\pause
\begin{proof}
Since any homogeneous system is consistent. We need only show there is a free variable. 
\bpause Let $m$ be the number of equations, or equivalently, the number of rows of the augmented matrix. 
By assumption $m<n$. 
\bpause Now the definition of a leading 1 implies that there can be \alert{at most} $m$ leading 1's, since there is at most one leading 1 per row.
\bpause This implies there are at most $m$ leading variables, and thus at least $n-m$ free variables. Since $n-m>0$ there is at least one free variable, and thus $\infty$-many solutions.   
\end{proof}
\end{frame}
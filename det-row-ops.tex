\begin{frame}{\ref{s:det}.\ref{ss:rowops}: row operations and the determinant}
We denote our various elementary row operations as 
\[
\rho_{c\cdot r_i}, \rho_{r_i\leftrightarrow r_j},\text{ and } \rho_{r_i+cr_j},
\]
as usual.
\pause
\begin{theorem}
Let $A$ be $n\times n$. Then:
\bb[(i)]
\pause\ii $\det(\rho_{c\cdot r_i}(A))=c\det(A)$,
\pause\ii $\det(\rho_{r_i\leftrightarrow r_j}(A))=-\det(A)$,
\pause\ii $\det(\rho_{r_i+cr_j}(A))=\det(A)$. 
\ee
\pause
Equivalently, we have: 
\bb[(i)]
\ii $\det(\underset{c\cdot r_i}{E}\cdot A)=c\det(A)$,
\ii $\det(\underset{r_i\leftrightarrow r_j}{E}\cdot A)=-\det(A)$,
\ii $\det(\underset{r_i+cr_j}{E}\cdot A)=\det(A)$. 
\ee
\end{theorem}
\end{frame}
\begin{frame}
\begin{theorem}\scriptsize
Let $A$ be $n\times n$. Then:
\bb[(i)]
\ii $\det(\underset{c\cdot r_i}{E}\cdot A)=c\det(A)$,
\ii $\det(\underset{r_i\leftrightarrow r_j}{E}\cdot A)=-\det(A)$,
\ii $\det(\underset{r_i+cr_j}{E}\cdot A)=\det(A)$. 
\ee
\end{theorem}
Setting $A=I_n$ in the last theorem, we get formulas for the determinant of an elementary matrix. 
\pause
\begin{corollary}
\bb[(i)]
\ii $\det(\underset{c\cdot r_i}{E})=c$,
\ii $\det(\underset{r_i\leftrightarrow r_j}{E})=-1$,
\ii $\det(\underset{r_i+cr_j}{E})=1$. 
\ee
\end{corollary}
\end{frame}
\begin{frame}
We can now use row operations to simplify the computation of $\det(A)$ as follows. We row reduce $A$ to some simpler matrix $B$, yielding a matrix equation of the form 
\[
E_rE_{r-1}\cdots E_1A=B.
\]
\pause
From our theorem above, it follows that 
\begin{eqnarray*}
\det(B)&=&\det(E_r)\det(E_{r-1}\cdots E_1A)\\
\pause&=&\det(E_r)\det(E_{r-1})\det(E_{r-2}\cdots E_1A)\\
\pause&=&\vdots \\
&=&\det(E_r)\det(E_r{-1})\cdots \det(E_1)\det(A)
\end{eqnarray*}
and thus that 
\[
\det(A)=\frac{\det(B)}{\det(E_r)\det(E_{r-1})\cdots \det(E_1)}.
\]
\end{frame}
\begin{frame}
As an example of this method we can easily prove the following corollary.
\begin{corollary}
Let $A$ be $n\times n$, and denote the rows of $A$ by $\boldr_i$. If $\boldr_i=c\boldr_j$ for some $i\ne j$ and some scalar $c$, then $\det(A)=0$. 

Similarly if $A$ has two columns that are scalar multiples of one another, then $\det(A)=0$.  
\end{corollary}
\pause
\begin{proof}
We assume $\boldr_i=c\boldr_j$ for some $i\ne j$ and some scalar $c$. 

\pause Using the row method of matrix multiplication, we see that the matrix  
\[
\tilde{A}=\underset{\boldr_i-cr_j}{E}\ \cdot A
\]
has a row of zeros in the $i$-th row. We conclude that $\det(\tilde{A})=0$. 

\pause
It follows that $0=\det(\underset{\boldr_i-cr_j}{E}\ \cdot A)=\det \underset{\boldr_i-cr_j}{E}\det A$. Since $\det \underset{\boldr_i-cr_j}{E}\ne 0$, we must have $\det A=0$. 

\pause 
The statement about columns follows from the statement about rows and the fact that $\det(A^T)=\det(A)$. 
\end{proof}
\end{frame}








\begin{frame}{\ref{s:det}.\ref{ss:further}: further properties of the determinant}
Our analysis of how the determinant changes with row operations leads to a number of very important theorems. 
\begin{theorem}[Determinant is multiplicative] \label{th:detmult} Let $A,B$ be $n\times n$. Then 
\[
\det(AB)=\det(A)\det(B).
\]
\end{theorem}
\pause\alert{Comment:} please observe, the same is NOT true of addition! 
\[
\det(A+B)\alert{\ne}\det(A)+\det(B).
\]
\pause
\begin{proof}
Your instructor will do this in class. 
\end{proof}
\end{frame}
\begin{frame}
\begin{theorem}[Determinant and invertibility]\label{th:detinv}
Let $A$ be $n\times n$. Then $A$ is invertible if and only if $\det(A)\ne 0$. 
\end{theorem}
%\alert{Comment:} thus we can add yet another equivalent statement to our Invertibility Theorem (see end of slides). 
\pause
\begin{proof} \ 

\alert{($\Rightarrow$)} \ I'll prove this direction as a chain of implications. 
\pause
\begin{align*}
A \text{ invertible}&\Rightarrow AA^{-1}=I_n \text{ for some matrix $A^{-1}$}\\
&\Rightarrow \det(AA^{-1})=\det I_n \\
&\Rightarrow \det(AA^{-1})=1 &\text{(since $\det I_n=1$)}\\
&\Rightarrow \det A\ \det A^{-1}=1 &\text{(by Th. \ref{th:detmult})}\\
&\Rightarrow \det A\ne 0 &\text{(otherwise product would be $0$)}
\end{align*} 
\pause
\alert{$(\Leftarrow)$} I'll prove the \alert{contrapositive} of this direction: that is, if $A$ is not invertible, then $\det A=0$. 
\pause 
Suppose $A$ is not invertible, and suppose $A$ reduces to the \alert{reduced} row echelon matrix $U$. By the invertibility theorem, $U$ is not the identity matrix. Thus one of its diagonal entries must be 0, and hence $\det U=0$. 
\bpause
This means we have 
$
E_rE_{r-1}\cdots E_1A=U
$, for some elementary matrices $E_i$, in which case $\det E_r\det E_{r-1} \cdots \det E_1\det A=\det U=0$, since $\det$ is multiplicative. \pause Since the $E_i$ are invertible, $\det E_i\ne 0$ (as shown above). It follows that $\det A=0$. 
\end{proof}
%\begin{theorem}
%The determinant is linear in each individual row, respectively each individual column, of a matrix.
%
%(Your professor will explain. A bit involved!) 
%\end{theorem}
\end{frame}
\begin{frame}{Adjoint formula for $A^{-1}$}
Its worth recording also that we can use the determinant to give an actual formula for computing the inverse of a matrix! 

We all love formulas, but keep in mind that it is faster to use our Gaussian elimination algorithm for computing $A^{-1}$.
\begin{theorem}[Adjoint formula]
Let $A$ be $n\times n$, and suppose $\det(A)\ne 0$. 

We define the {\bf adjoint matrix} of $A$ as 
\[
\adjoint(A)=\left([C_{ij}]\right)^T;
\]
that is, the $ij$-th entry of the adjoint matrix is the $ji$-th cofactor of $A$. 

Then 
\[
A^{-1}=\frac{1}{\det(A)}\adjoint(A).
\]

\end{theorem}
\pause
\begin{proof}[Proof sketch] Let $B=\adjoint A$, and let $c=\det A$. Using the definition of $\adjoint A$ and properties of the determinant, you can in fact show directly that 
\[
AB=c I_n.
\]
It follows that $A^{-1}=\frac{1}{c}B=\frac{1}{\det A}\adjoint A$.
\end{proof}
\end{frame}
\begin{frame}{Our growing invertibility theorem}
Thanks to Theorem \ref{th:detinv}, our invertibility theorem has grown by one statement. 
\begin{theorem}[Invertibility theorem]
Let $A$ be $n\times n$. The following statements are equivalent. 
\bb[(a)]
\ii $A$ is invertible.
\ii $A\boldx=\boldzero$ has a unique solution (the trivial one). 
\ii $A$ is row equivalent to $I_n$, the $n\times n$ identity matrix.
\ii $A$ is a product of elementary matrices. 	
\ii $A\boldx=\boldb$ has a solution for every $n\times 1$ column vector $\boldb$. 
\ii $A\boldx=\boldb$ has a {\em unique} solution for every $n\times 1$ column vector $\boldb$. 
\ii $\det(A)\ne 0$.
\ee
\end{theorem}

\end{frame}
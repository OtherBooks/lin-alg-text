In what follows $V$ is always a vector space along with an inner product $\langle \ , \rangle$. Notions of norm and orthogonality always refer to this fixed choice of inner product. 
\bb
\ii Show that $\frac{\boldv}{\norm{\boldv}}:=\frac{1}{\norm{\boldv}}\cdot\boldv$ is a unit vector for any $\boldv\in V$. 
%\vfill
\\
\begin{solution}
In any inner product space the norm operation satisfies $\norm{c\boldv}=\val{c}\norm{\boldv}$. (Observe the subtle notational difference: $\val{c}$ is absolute value of the real number $c$; $\norm{\boldv}$ is norm of the vector $\boldv$. )

Now let $\ds\boldw=\frac{1}{\norm{\boldv}}\boldv$. Then 
\begin{align*}
\norm{\boldw}&=\norm{\frac{1}{\norm{\boldv}}\boldv}\\
&=\val{\frac{1}{\norm{\boldv}}}\norm{\boldv}\\
&=\frac{1}{\norm{\boldv}}\norm{\boldv} &\text{(since $\ds\frac{1}{\norm{\boldv}}\geq 0$)}\\
&=1
\end{align*}
\end{solution}

\ii Let $(V, \angvec{ \ , })$ be an inner product space. Prove: $\norm{c\boldv}=\val{c}\norm{\boldv}$ for all $c\in \R$. 

(Here $\val{c}$ is the absolute value of $c$.) 

Your proof must be valid for a general inner product space; i.e., you can only invoke the definition of $\norm{ \underline{\hspace{5pt}}}$ in terms of $\angvec{\ , }$, the axioms of an inner product, and properties of real number arithmetic. 
\\
\begin{solution}
 \begin{align*}
  \norm{c\boldv}^2&=\angvec{c\boldv, c\boldv}\\
  &=c^2\angvec{\boldv, \boldv}\\
  &=c^2\norm{\boldv}^2
 \end{align*} 
 This implies 
 \[
 \norm{c\boldv}=\sqrt{c^2}\norm{\boldv}=\val{c}\norm{\boldv}.
 \]
\end{solution} 


\ii Compute the angle between $\boldv=(1,1,1,1)$ and $\boldw=(1,-1,1,1)$. Do NOT use a calculator. 
%\vfill
\begin{solution}
The equation for the angle $\theta$ ends up being $\cos(\theta)=\frac{1}{2}$. The unique $\theta$ in the range $[0,\pi]$ satisfying this is $\theta=\pi/3=60^\circ$.
\end{solution}
\ii Let $V=C([0,1])$ with inner product $\ds \angvec{ f, g}=\int_0^1f(x)g(x) \ dx$. Let $f(x)=1$ and $g(x)=x$. Compute the angle $\theta$ between $f$ and $g$ with respect to this inner product. 

Your answer cannot be expressed in terms of inverse trig functions: i.e., the angle is a familiar one that you can solve for by hand. 
\\
\begin{solution}
\noindent
The angle $\theta$ is the unique angle between 0 and $\pi$ satisfying 
\[
\cos(\theta)=\frac{\angvec{f,g}}{\norm{f}\norm{g}}=\frac{\int_0^1x \ dx}{\sqrt{\int_0^11\ dx}\sqrt{\int_0^1 x^2 \ dx}}=\frac{1/2}{1/\sqrt{3}}=\frac{\sqrt{3}}{2}.
\] 
Thus $\theta=\pi/6$. 
\end{solution} 
%\newpage
\ii Prove that if $\boldw$ is orthogonal to $\boldv_1,\boldv_2,\dots, \boldv_n$, then it is orthogonal to any linear combination of the $\boldv_i$. In other words prove the implication:
\[
(\boldw\perp\boldv_i \text{ for all $i$})\Rightarrow (\boldw\perp(c_1\boldv_1+c_2\boldv_2+\cdots+c_n\boldv_n) \text{ for any $c_i$})
\]
%\vfill
\begin{solution}
We prove both directions of the if and only if separately. 

$(\Leftarrow)$: this is the easy direction. If $\boldw$ is orthogonal to {\em any} linear combination of the $\boldv_i$, then in particular it is orthogonal to each of the $\boldv_i$. 

($\Rightarrow)$: assume $\boldw\perp\boldv_i$ for all $i$. Then $\boldw\cdot\boldv_i=0$ for all $i$. But then 
\begin{align*}
\boldw\cdot(c_1\boldv_1+\cdots +c_n\boldv_n)&=\boldw\cdot(c_1\boldv_1)+\cdots +\boldw\cdot(c_n\boldv_n) &\text{(add. prop. of dot)}\\
&=c_1(\boldw\cdot\boldv_1)+\cdots +c_n\boldw\cdot\boldv_n) &\text{(scalar mult. prop. of dot)}\\
&=0+0+\cdots +0=0.
\end{align*}
Thus $\boldw\perp(c_1\boldv_1+\cdots +c_n\boldv_n)$ for any $c_i$, as desired. 
\end{solution}

%\newpage
\ii For each of the following operations on $\R^2$, determine whether it defines an inner product on $\R^2$. If it fails to be an inner product, identify which of the three inner product axioms (if any) it does satisfy, and  provide explicit counterexamples for any axiom that fails.
\bb
\ii $\angvec{(x_1,x_2),\ (y_1,y_2)}=x_1y_2+x_2y_1$. 
\ii $\angvec{(x_1,x_2),\ (y_1,y_2)}=2x_1y_1+x_1y_2+x_2y_1+3x_2y_2$. 
\ii $\angvec{(x_1,x_2), \ (y_1,y_2)}=x_1^2y_1^2+x_2^2y_2^2$.
\ee
\begin{solution}
\noindent
In what follows we let $\boldx=(x_1,x_2)$, $\boldy=(y_1,y_2)$, $\boldz=(z_1,z_2)$. 

(a) The proposed operation satisfies all axioms except positivity. Let's see why. 
\\
Symmetry. We have 
\begin{align*}
\angvec{\boldy,\ \boldx}&=y_1x_2+y_2x_1\\
&=x_1y_2+x_2y_1\\
&=\angvec{\boldx, \boldy}
\end{align*}
Linearity in first variable. We have 
\begin{align*}
\angvec{c\boldx+d\boldy,\boldz}&=\angvec{(cx_1+dy_1,cx_2+dy_2),\ (z_1,z_2)}\\
&=(cx_1+dy_1)z_2+(cx_2+dy_2)z_1\\
&=c(x_1z_2+x_2z_1)+d(y_1z_2+y_2z_1)\\
&=c\angvec{\boldx,\ \boldy}+d\angvec{\boldx, \ \boldz}
\end{align*}
Positivity. This fails. Take $\boldx=(1,-1)$. Then 
\[
\angvec{\boldx,\boldx}=1\cdot (-1)+(-1)\cdot 1=-2<0.
\]
Furthermore, taking $\boldx=(0,1)$, we see that $\angvec{\boldx,\boldx}=0$. Thus both parts of positivity fail. 
\vspace{.1in}
\\
(b) This is in fact an inner product! The proofs that properties  (i) and (ii) hold proceed exactly as the argument above. Let's focus on positivity. We have 
\begin{align*}
\angvec{\boldx,\boldx}&=2x_1^2+2x_1x_2+3x_2^2\\
&=2(x_1+\frac{1}{2}x_2)^2+\frac{11}{4}x_2^2 &\text{(complete the square)}\\
&=2u^2+\frac{11}{4}v^2 &(u=x_1+\frac{1}{2}x_2, \ v=x_2)
\end{align*}
Clearly the expression $2u^2+\frac{11}{4}v^2$ is nonnegative. Furthermore this will be equal to zero if and only if $u=v=0$ if and only if $x_1+\frac{1}{2}{x_2}=x_2=0$ if and only if $x_1=x_2=0$. This proves $\angvec{\boldx,\boldx}\geq 0$ and $\angvec{\boldx, \boldx}=0$ if and only if $\boldx=\boldzero$. 
\vspace{.1in}
\\
(c) It is clear that the operation is symmetric: that is $\angvec{\boldx, \boldy}=\angvec{\boldy,\boldx}$. 
Thus axiom (i) is satsified. 

\noindent
The operation does not satisfy axiom (ii). For example, Let $\boldx=(1,1)$, $\boldy=(0,1)$. Then 
\[
\angvec{2\boldx,\boldy}=\angvec{(2,2),(0,1)}=4\ne 2\angvec{\boldx,\boldy}=1.
\]
The operation does satisfy positivity, as is easily shown. 
\end{solution}
\ii Equip $P_2$ with the {\em evaluation inner product}  $\angvec{p,q}=p(-1)q(-1)+p(0)q(0)+p(1)q(1)$. 
\\
Find all polynomials $p(x)$ orthogonal to $q(x)=x$ with respect to this inner product. 
\\
Note: this is an infinite set of polynomials. Describe it by giving a parameter description of its elements.
\\
\begin{solution}
\noindent A polynomial $p$ is orthogonal to $q$ if and only if $p(-1)q(-1)+p(0)q(0)+p(1)q(1)=0$ if and only if $-p(-1)+p(1)=0$. If $p(x)=ax^2+bx+c$, this means $-(a-b+c)+(a+b+c)=0$, or $2b=0$. Thus $p(x)=ax^2+c$, for $a,c\in\R$ any real numbers. 
\end{solution} 
\ii Equip $V=C([-\pi,\pi])$ with the {\em integral inner product} $\angvec{f, g}=\int_{-\pi}^\pi f(x)g(x) \ dx$.
\bb
\ii Show that $f(x)=\cos(x)$ and $g(x)=\sin(x)$ are orthogonal with respect to this inner product. 
\ii Compute $\norm{\cos(x)}$ with respect to this inner product. 
\ii Show that if $f(x)$ is an odd function (i.e., $f(x)=-f(-x)$ for all $x$) and $g(x)$ is an even function ($g(-x)=g(x)$ for all $x$), then $f$ and $g$ are orthogonal with respect to this inner product.  
\\
{\bf Hint}: use the area interpretation of the integral, as well as graphical properties of even/odd functions. 
\ee 
\begin{solution}
\noindent 
Orthogonality follows from $\int_{-\pi}^\pi \cos x\sin x\ dx=\int_{\pi}^{\pi}\frac{1}{2}\sin(2x)\ dx=0$, as one easily shows. 

We have 
\[
\norm{\cos x}=\sqrt{\angvec{\cos x, \cos x}}=\sqrt{\int_{-\pi}^\pi \cos^2(x) \ dx}=\sqrt{\pi}. 
\]

If $f$ is odd and $g$ is even, then their product $fg$ is odd. The integral of an odd function over a symmetric interval like $[-\pi, \pi]$ is always 0. You can see this using the area interpretation (any region to the right of the $y$-axis has a flipped region to the left that cancels out its contribution to the integral), or by using integral properties:
\begin{align*}
\int_{-\pi}^\pi f(x)g(x) \ dx&=\int_{\pi}^{-\pi} -f(-u)g(-u) \ du &\text{(change of variables $u=-x, du=-dx$)}\\
&= \int_{\pi}^{-\pi} f(u)g(u) \ du&\text{(since $fg$ is odd)}\\
&=-\int_{-\pi}^\pi f(x)g(x) \ dx &\text{(flip the enpoints, change variable name)}
\end{align*}
Since $\int_{-\pi}^\pi f(x)g(x) \ dx=-\int_{-\pi}^\pi f(x)g(x) \ dx$, we conclude $\angvec{f, g}=\int_{-\pi}^\pi f(x)g(x) \ dx=0$. 
\end{solution}
\ii Let $\boldv, \boldw\in V$, and let $\theta$ be the angle between them. Prove the following equivalence:
\[
\norm{\boldv+\boldw}=\norm{\boldv}+\norm{\boldw}\text{ if and only if } \theta=0.
\]
Your proof should be a {\em chain of equivalences} with each step justified. 
\\
\begin{solution}
\begin{align*}
\norm{\boldv+\boldw}=\norm{\boldv}+\norm{\boldw}&\Leftrightarrow
\norm{\boldv+\boldw}^2=\left(\norm{\boldv}+\norm{\boldw}\right)^2
&\text{( square both sides)}\\
&\Leftrightarrow (\boldv+\boldw)\cdot(\boldv+\boldw)=\norm{\boldv}^2+2\norm{\boldv}\norm{\boldw}+\norm{\boldw}^2\\
&\Leftrightarrow \boldv\cdot\boldv+2\boldv\cdot\boldw+\boldw\cdot\boldw=\boldv\cdot\boldv+2\norm{\boldv}\norm{\boldw}+\boldw\cdot\boldw\\
&\Leftrightarrow \boldv\cdot\boldw=\norm{\boldv}\norm{\boldw}\\
&\Leftrightarrow \frac{\boldv\cdot\boldw}{\norm{\boldv}\norm{\boldw}}=1\\
&\Leftrightarrow \cos(\theta)=1\\
&\Leftrightarrow \theta=0.
\end{align*}
\end{solution}
\ii Take $\boldv,\boldw\in V$. Suppose $\norm{\boldv}=2$ and $\norm{\boldw}=3$. Find the maximum and minimum possible values of $\norm{\boldv-\boldw}$, and give explicit examples where those values occur. 
%\vfill
\\
\begin{solution}
We  we have 
\begin{align*}
\norm{\boldv-\boldw}^2&=(\boldv-\boldw)\cdot(\boldv-\boldw)\\
&=\norm{\boldv}^2-2\boldv\cdot\boldw+\norm{\boldw}^2.
\end{align*}
By Cauchy-Schwarz 
\[
- \norm{\boldv}\norm{\boldw}\leq\boldv\cdot\boldw\leq \norm{\boldv}\norm{\boldw}.
\]
Thus 
\[
\norm{\boldv}^2-2\norm{\boldv}\norm{\boldw}+\norm{\boldw}^2\leq\norm{\boldv-\boldw}^2\leq 
\norm{\boldv}^2+2\norm{\boldv}\norm{\boldw}+\norm{\boldw}^2
\]
which implies 
\[
(\norm{\boldv}-\norm{\boldw})^2\leq\norm{\boldv-\boldw}^2\leq(\norm{\boldv}+\norm{\boldw})^2
\]
and thus
\[
\norm{\boldv}-\norm{\boldw}\leq\norm{\boldv-\boldw}\leq\norm{\boldv}+\norm{\boldw}
\]
So in our example 
\[
1\leq \norm{\boldv-\boldw}\leq 5.
\]
In the special case when $V=\R^2$ with the dot product, the picture of this situation consists of two circles centered about the origin: one of radius 2, the other of radius 3. The vector $\boldv$ represents a point on the small circle; the vector $\boldw$ represents a point on the large one. These are closest when they lie along the same radius vector from the origin; they are furthest when lying on diametrically opposed radii.  
\end{solution}
\ii Prove the {\em Pythagorean theorem for general inner product spaces}: if $\boldv\perp\boldw$, then 
\[
\norm{\boldv+\boldw}^2=\norm{\boldv}^2+\norm{\boldw}^2.
\]
\begin{solution}
We have 
\begin{align*}
\norm{\boldv+\boldw}^2&=\langle\boldv+\boldw,\boldv+\boldw\rangle\\
&=\langle\boldv,\boldv\rangle+2\cancelto{0}{\langle\boldv,\boldw\rangle}+\langle\boldw,\boldw\rangle &\text{(since $\boldv\perp\boldw$)}\\
&=\norm{\boldv}^2+\norm{\boldw}^2
\end{align*}
\end{solution}
\ii Prove each inequality below using a judicious choice of inner product in conjunction with the Cauchy-Schwarz inequality (and possibly a judicious choice of one of the vectors in the Cauchy-Schwarz inequality). 
\bb
\ii  For all $f, g\in C([a,b])$
\[
\left(\int_a^b f(x)g(x) \ dx\right)^2\leq \int_a^b f^2(x)\ dx\int_a^b g^2(x) \ dx.\]
\ii For all $(x_1,x_2,\dots, x_n)\in\R^n$,
\[
(x_1+x_2+\cdots +x_n)\leq\sqrt{x_1^2+x_2^2+\cdots +x_n^2}\sqrt{n}. 
\]
\ii For all $a,b,\theta\in\R$ 
\[
(a\cos(\theta)+b\sin(\theta))^2\leq a^2+b^2.
\]

\ee
\begin{solution}
\noindent 
(a) Once we translate the integrals as inner products, this inequality is obtained by squaring both sides of the Cauchy-Schwarz (C-S) inequality. 
\\
(b) This is an instance of C-S, where $V=\R^n$ with dot product, $\boldx$ is arbitrary, and $\boldy=(1,1,\dots, 1)$. 
\\
(c) This is an instance of C-S, where $V=\R^2$ with dot product, $\boldx=(a,b)$ and $\boldy=(\cos(\theta), \sin(\theta))$. Note that $\norm{\boldy}=1$ for this choice of $\boldy$. 
\end{solution}
\begin{samepage}
\ii Let $(V,\angvec{ \ , })$ be an inner product space. Recall in this context that we define $\norm{\boldv}=\sqrt{\angvec{\boldv, \boldv}}$, and $d(\boldv, \boldw)=\norm{\boldv-\boldw}$. 
\\
An {\bf isometry}  of $V$ is a function $f\colon V\rightarrow V$ that preserves distance: i.e., 
\[
d(f(\boldv), f(\boldw))=d(\boldv, \boldw) \text{ for all $\boldv, \boldw\in V$}.
\]
In this exercise we will show that any isometry that maps $\boldzero$ to $\boldzero$ is a linear transformation. \\
(This is a handy fact to know. For example, reflections through lines in $\R^2$ and planes in $\R^3$ clearly preserve distances and map $\boldzero$ to itself; it follows immediately that these operations are linear. )

\noindent
In what follows, assume that $f$ is an isometry of $V$ satisfying $f(\boldzero)=\boldzero$.  
\bb
\ii Prove that $\norm{f(\boldv)}=\norm{\boldv}$: i.e., $f$ preserves norms. 
\ii Prove $\angvec{f(\boldv), f(\boldw)}=\angvec{\boldv, \boldw}$: i.e., $f$ preserves inner products. 
\\
Hint: first prove that $\angvec{\boldv, \boldw}=\frac{1}{2}(\norm{\boldv}^2+\norm{\boldw}^2-\norm{\boldv-\boldw}^2)$.
\ii To prove $f$ is linear it is enough to show $f(\boldv+c\boldw)=f(\boldv)+cf(\boldw)$ for all $\boldv, \boldw\in V$, $c\in \R$. \\
To do so, use the above parts to show that 
\[
\norm{f(\boldv+c\boldw)-(f(\boldv)+cf(\boldw))}=0.
\]  
\noindent
Hint: regroup this difference in a suitable manner so that you can use parts (a)-(b). You may also want to use the identity 
\[
\norm{\boldv-\boldw}^2=\norm{\boldv}^2-2\angvec{\boldv,\boldw}+\norm{\boldw}^2.
\]
\ee
\end{samepage}
\begin{solution}
\noindent
(a) We have 
\begin{align*}
\norm{f(\boldv)}&=\norm{f(\boldv)-\boldzero}\\
&=\norm{f(\boldv)-f(\boldzero)} &\text{(since $f(\boldzero)=\boldzero$)}\\
&=d(f(\boldv),f(\boldzero) &\text{(by def.)}\\
&=d(\boldv, \boldzero) &\text{($f$ is an isometry)}\\
&=\norm{\boldv}
\end{align*}
(b) The hint here is proved by starting with the RHS and substituting $\norm{\boldv}^2=\angvec{\boldv, \boldv}, \norm{\boldw}^2=\angvec{\boldw,\boldw}, \norm{\boldv-\boldw}^2=\angvec{\boldv-\boldw,\boldv-\boldw}$, and then using algebraic properties of the inner product to simplify the resulting expression until you get the LHS. 

Once we have shown the equality mentioned in the hint, the rest is easy:
\begin{align*}
\angvec{f(\boldv),f(\boldw)}&=\frac{1}{2}(\norm{f(\boldv)}^2+\norm{f(\boldw)}^2-\norm{f(\boldv)-f(\boldw)}^2) &\text{(hint equality)}\\
&=\frac{1}{2}(\norm{f(\boldv)}^2+\norm{f(\boldw)}^2-d(f(\boldv), f\boldw))^2) &\text{(def. of dist.)}\\
&=\frac{1}{2}(\norm{\boldv}^2+\norm{\boldw}^2-d(\boldv, \boldw)^2) &\text{($f$ preserves norm and dist.)}\\
&=\frac{1}{2}(\norm{\boldv}^2+\norm{\boldw}^2-\norm{\boldv-\boldw}^2) &\text{(def. of dist.)}\\
&=\angvec{\boldv,\boldw} &\text{(hint equality)}
\end{align*}
(c) Note that proving 
\[
\norm{f(\boldv+c\boldw)-(f(\boldv)+cf(\boldw))}=0
\]
implies that $f(\boldv+c\boldw)-(f(\boldv)+cf(\boldw))=\boldzero$, and hence that $f(\boldv+c\boldw)=f(\boldv)+cf(\boldw)$, as desired. We will prove the square of this norm is 0, and we group the expression as $(f(\boldv+c\boldw)-f(\boldv))-cf(\boldw)$. 
\\
We have
\begin{align*}
&\norm{\left(f(\boldv+c\boldw)-f(\boldv)\right)-cf(\boldw))}^2\\
&=\norm{f(\boldv+c\boldw)-f(\boldv)}^2-2\angvec{f(\boldv+c\boldw)-f(\boldv),cf(\boldw)}+\norm{cf(\boldw)}^2 &\text{(suggested identity)}\\
&=\norm{f(\boldv+c\boldw)-f(\boldv)}^2-2c\angvec{f(\boldv+c\boldw)-f(\boldv),f(\boldw)}+c^2\norm{f(\boldw)}^2 \\
&=\norm{\boldv+c\boldw- \boldv}^2-2c\angvec{f(\boldv+c\boldw)-f(\boldv),f(\boldw)}+c^2\norm{\boldw}^2 &\text{($f$ preserves distance and norms)}\\
&=c^2\norm{\boldw}^2-2c\angvec{f(\boldv+c\boldw),f(\boldw)}+2c\angvec{f(\boldv),f(\boldw)}+c^2\norm{\boldw}^2\\
&=c^2\norm{\boldw}^2-2c\angvec{\boldv+c\boldw,\boldw}+2c\angvec{\boldv, \boldw} +c^2\norm{\boldw}^2 &\text{$f$ preserves inner product)}\\
&\vdots &\text{(inner product simplification)}\\
&=0
\end{align*}
\end{solution}
\ee
